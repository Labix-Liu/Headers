%=========================================
% Theorem Environment
%=========================================
%-------------------------------------------------------------------------
% General Box Settings
%-------------------------------------------------------------------------

\tcbset{
    sticky/.style={
            bicolor,
            underlay={
                    \begin{tcbclipframe}
                        \draw [line width=1mm, color=#1] (interior.south west)--(interior.north west);
                    \end{tcbclipframe}
                },
            pad at break*=2mm, breakable, boxrule=0mm,
            left=4mm, right=4mm, top=1mm, bottom=1mm,
            frame hidden, colback=#1!8, colbacklower=#1!2,
            arc=.5mm, beforeafter skip balanced=3mm,
            detach title, fonttitle=\bfseries,
            description delimiters parenthesis,
            description font=\mdseries,
            separator sign none, terminator sign none,
            coltitle=#1!70!black,
            before upper={\tcbtitle\quad}
        },
    titled/.style={enhanced, pad at break*=2mm, breakable,
            left=4mm, right=4mm, top=1mm, bottom=1mm,
            colback=#1!8, boxrule=0.5mm,
            colframe=#1, arc=.5mm,
            fonttitle=\bfseries, coltitle=#1!8,
            beforeafter skip balanced=3mm
        },
    boxmath/.style={
            left=4mm, right=4mm, top=1mm, bottom=1mm,
            colback=DarkBlue!8, boxrule=0.5mm,
            colframe=DarkBlue, arc=.5mm,
            fonttitle=\bfseries, coltitle=DarkBlue!8,
        }
}

%-------------------------------------------------------------------------
% Tcb Environments
%-------------------------------------------------------------------------

\newtcbtheorem[number within=subsection]{thm}
{Theorem}{
	sticky=EarthYellow, 
	label type=theorem
}{thm}

\newtcbtheorem[number within=subsection, use counter from=thm]{prp}
{Proposition}{
	sticky=RoyalPurple, 
	label type=proposition
}{prp}

\newtcbtheorem[number within=subsection, use counter from=thm]{conj}
{Conjecture}{
	sticky=Royal, 
	label type=conjecture
}{conj}

\newtcbtheorem[number within=subsection, use counter from=thm]{crl}
{Corollary}{
	sticky=Royal, 
	label type=corollary
}{crl}

\newtcbtheorem[number within=subsection, use counter from=thm]{lmm}
{Lemma}{
	sticky=Royal, 
	label type=lemma
}{lmm}

\newtcbtheorem[number within=subsection, use counter from=thm]{defn}
{Definition}{
	sticky=ForestGreen, 
	label type=definition
}{defn}

\newtcbtheorem[number within=subsection, use counter from=thm]{eg}
{Example}{
	sticky=NavyBlue, 
	label type=example
}{eg}

\newtcbtheorem[number within=subsection, use counter from=thm]{ex}
{Exercise}{
	sticky=PastelPink, 
	label type=exercise
}{ex}

\crefname{exercise}{exercise}{exercises}
\Crefname{exercise}{Exercise}{Exercises}

\newtcolorbox[
    auto counter,
    number freestyle={\noexpand\arabic{\tcbcounter})}]
{question}{
    detach title,
    title={Question \thetcbcounter},
    coltitle=black,
    fonttitle=\bfseries,
    before upper={\tcbtitle\quad},
}

%-------------------------------------------------------------------------
% Tcb Environments Without Counters
%-------------------------------------------------------------------------

\newtcbtheorem[no counter]{note}
{Note}{sticky=Goldenrod, label type=note}{note}
\newtcbtheorem[number within=section]{remark}
{Remark}{sticky=Goldenrod, label type=remark}{rem}
\newtcbtheorem[number within=section]{recall}
{Recall}{sticky=Goldenrod, label type=recall}{recall}
\newtcbtheorem[no counter]{property}
{Property}{sticky=Goldenrod, label type=property}{propty}
\newtcbtheorem[no counter]{warning}
{Warning}{sticky=BrickRed, label type=warning}{warn}

\newtcbtheorem[no counter]{notation}
{Notation}{sticky=Gray, label type=notation}{notation}

%-------------------------------------------------------------------------
% Formulas
%-------------------------------------------------------------------------

\newtcolorbox{formulainner}[2]{
    titled=DarkBlue,
    title=#1, label=#2,
    label type=formula
}

\ExplSyntaxOn
\NewDocumentEnvironment{formula}{O{}}
{
    \keys_set:nn { form/tcb } { #1 }
    \form_tcb_begin:nVV {formulainner} \l__form_tcb_title_tl \l__form_tcb_label_tl
}
{
    \end{formulainner}
}

\cs_new_protected:Nn \form_tcb_begin:nnn
{
    \begin{#1}{#2}{#3}
}
\cs_generate_variant:Nn \form_tcb_begin:nnn { nVV }
\keys_define:nn { form/tcb }
{
    title .tl_set:N = \l__form_tcb_title_tl,
    label .tl_set:N = \l__form_tcb_label_tl,
}
\ExplSyntaxOff
\crefname{formula}{formula}{formulas}
\Crefname{formula}{Formula}{Formulas}

\NewDocumentCommand{\boxmath}{O{} m}
{
    \tcboxmath[boxmath, #1]{#2}
}

%-------------------------------------------------------------------------
% Proof and Answer Environment
%-------------------------------------------------------------------------

% If the proof/answer environment is inside a theorem environment then we don't wrap with a tcolorbox and instead split the current tcolorbox
\newbool{intheorem}

\NewDocumentCommand{\patchtheorem}{mms}{
    \AtBeginEnvironment{#1}{\colorlet{proofcolor}{#2}\booltrue{intheorem}}
    \AfterEndEnvironment{#1}{\colorlet{proofcolor}{#2}}
    {
        \IfBooleanT{#3}{
            \AtBeginEnvironment{#1*}{\colorlet{proofcolor}{#2}\booltrue{intheorem}}
            \AfterEndEnvironment{#1*}{\colorlet{proofcolor}{#2}}

        }
    }
}

% Default proofcolor
\colorlet{proofcolor}{Royal}
\patchtheorem{thm}{EarthYellow}*
\patchtheorem{prp}{RoyalPurple}*
\patchtheorem{conjecture}{Royal}*
\patchtheorem{crl}{Royal}*
\patchtheorem{lmm}{Royal}*
\patchtheorem{defn}{ForestGreen}*
\patchtheorem{eg}{NavyBlue}*
\patchtheorem{ex}{PastelPink}*
\patchtheorem{remark}{Goldenrod}

\renewcommand{\qed}{\null\hfill\textcolor{proofcolor}{\(\blacksquare\)}}

\newenvironment{wraptheorem}{
    \ifbool{intheorem}{\vspace{-4.5pt}}{\begin{tcolorbox}[
                sticky=proofcolor,
                colback=proofcolor!1, before upper={}
            ]}}
            {\ifbool{intheorem}{}{\end{tcolorbox}}}

\makeatletter
\renewenvironment{proof}[1][Proof]{
    \begin{wraptheorem}\par
        \pushQED{\qed}%
        \normalfont \topsep6\p@\@plus6\p@\relax
        \trivlist
        \item{\color{proofcolor!70!black}\sffamily\bfseries
                    #1}\ignorespaces\hspace{0.8em}
        \setlength{\parskip}{\baselineskip/2}
        }{%
        \popQED\endtrivlist\@endpefalse\end{wraptheorem}
}
\makeatother

\newenvironment{answer}[1][Answer]{
    \renewcommand{\qed}{\null\hfill\textcolor{proofcolor}{\(\circledast\)}}
    \proof[#1]}{\endproof}

\BeforeBeginEnvironment{proof}{\ifbool{intheorem}{\tcblower}{}}
\BeforeBeginEnvironment{answer}{\ifbool{intheorem}{\tcblower}{}}

\crefformat{proof}{#2proof#3}
\crefname{answer}{answer}{answers}
\Crefname{answer}{Answer}{Answers}

% ============================================================================
% Hide theorem environments
% ============================================================================
\usepackage{environ}
\NewEnviron{killcontents}{}
\newbool{hide_exercises}
\AfterEndPreamble{\ifbool{hide_exercises}{
        \RenewDocumentEnvironment{exercise}{o}{\killcontents}{\endkillcontents}
    }{}}