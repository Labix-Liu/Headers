%=========================================
% Theorem Environment
%=========================================
% correct
\definecolor{correct}{HTML}{009900}
\newcommand\correct[2]{\ensuremath{\:}{\color{red}{#1}}\ensuremath{\to }{\color{correct}{#2}}\ensuremath{\:}}
\newcommand\green[1]{{\color{correct}{#1}}}

% hide parts
\newcommand\hide[1]{}

% si unitx
\usepackage{siunitx}
\sisetup{locale = FR}
% \renewcommand\vec[1]{\mathbf{#1}}
\newcommand\mat[1]{\mathbf{#1}}

% Clear out the item autoref name
\makeatletter
\def\itemautorefname{\@gobble}
\makeatother

% theorems
\makeatother
\usepackage{thmtools}
\usepackage[framemethod=TikZ]{mdframed}

\mdfsetup{skipabove=1em,skipbelow=0em}

\theoremstyle{definition}

\declaretheoremstyle[
	headfont=\bfseries\sffamily\color{ForestGreen!70!black}, bodyfont=\normalfont,
	mdframed={
			linewidth=2pt,
			rightline=false, topline=false, bottomline=false,
			linecolor=ForestGreen, backgroundcolor=ForestGreen!5,
			nobreak=false
		}
]{thmgreenbox}

\declaretheoremstyle[
	headfont=\bfseries\sffamily\color{ForestGreen!70!black}, bodyfont=\normalfont,
	mdframed={
			linewidth=2pt,
			rightline=false, topline=false, bottomline=false,
			linecolor=ForestGreen, backgroundcolor=ForestGreen!8,
			nobreak=false
		}
]{thmgreen2box}

\declaretheoremstyle[
	headfont=\bfseries\sffamily\color{NavyBlue!70!black}, bodyfont=\normalfont,
	mdframed={
			linewidth=2pt,
			rightline=false, topline=false, bottomline=false,
			linecolor=NavyBlue, backgroundcolor=NavyBlue!5,
			nobreak=false
		}
]{thmbluebox}

\declaretheoremstyle[
	headfont=\bfseries\sffamily\color{TealBlue!70!black}, bodyfont=\normalfont,
	mdframed={
			linewidth=2pt,
			rightline=false, topline=false, bottomline=false,
			linecolor=TealBlue,
			nobreak=false
		}
]{thmblueline}

\declaretheoremstyle[
	headfont=\bfseries\sffamily\color{RawSienna!70!black}, bodyfont=\normalfont,
	mdframed={
			linewidth=2pt,
			rightline=false, topline=false, bottomline=false,
			linecolor=RawSienna, backgroundcolor=RawSienna!5,
			nobreak=false
		}
]{thmredbox}

\declaretheoremstyle[
	headfont=\bfseries\sffamily\color{RawSienna!70!black}, bodyfont=\normalfont,
	mdframed={
			linewidth=2pt,
			rightline=false, topline=false, bottomline=false,
			linecolor=RawSienna, backgroundcolor=RawSienna!8,
			nobreak=false
		}
]{thmred2box}

\declaretheoremstyle[
	headfont=\bfseries\sffamily\color{SeaGreen!70!black}, bodyfont=\normalfont,
	mdframed={
			linewidth=2pt,
			rightline=false, topline=false, bottomline=false,
			linecolor=SeaGreen, backgroundcolor=SeaGreen!2,
			nobreak=false
		}
]{thmgreen3box}

\declaretheoremstyle[
	headfont=\bfseries\sffamily\color{WildStrawberry!70!black}, bodyfont=\normalfont,
	mdframed={
			linewidth=2pt,
			rightline=false, topline=false, bottomline=false,
			linecolor=WildStrawberry, backgroundcolor=WildStrawberry!5,
			nobreak=false
		}
]{thmpinkbox}

\declaretheoremstyle[
	headfont=\bfseries\sffamily\color{MidnightBlue!70!black}, bodyfont=\normalfont,
	mdframed={
			linewidth=2pt,
			rightline=false, topline=false, bottomline=false,
			linecolor=MidnightBlue, backgroundcolor=MidnightBlue!5,
			nobreak=false
		}
]{thmblue2box}

\declaretheoremstyle[
	headfont=\bfseries\sffamily\color{Gray!70!black}, bodyfont=\normalfont,
	mdframed={
			linewidth=2pt,
			rightline=false, topline=false, bottomline=false,
			linecolor=Gray, backgroundcolor=Gray!5,
			nobreak=false
		}
]{notgraybox}

\declaretheoremstyle[
	headfont=\bfseries\sffamily\color{Gray!70!black}, bodyfont=\normalfont,
	mdframed={
			linewidth=2pt,
			rightline=false, topline=false, bottomline=false,
			linecolor=Gray,
			nobreak=false
		}
]{notgrayline}

% \declaretheoremstyle[
% 	headfont=\bfseries\sffamily\color{RawSienna!70!black}, bodyfont=\normalfont,
% 	numbered=no,
% 	mdframed={
% 			linewidth=2pt,
% 			rightline=false, topline=false, bottomline=false,
% 			linecolor=RawSienna, backgroundcolor=RawSienna!1,
% 		},
% 	qed=\qedsymbol
% ]{thmproofbox}

\declaretheoremstyle[
	headfont=\bfseries\sffamily\color{NavyBlue!70!black}, bodyfont=\normalfont,
	numbered=no,
	mdframed={
			linewidth=2pt,
			rightline=false, topline=false, bottomline=false,
			linecolor=NavyBlue, backgroundcolor=NavyBlue!1,
			nobreak=false
		}
]{thmexplanationbox}

\declaretheoremstyle[
	headfont=\bfseries\sffamily\color{WildStrawberry!70!black}, bodyfont=\normalfont,
	numbered=no,
	mdframed={
			linewidth=2pt,
			rightline=false, topline=false, bottomline=false,
			linecolor=WildStrawberry, backgroundcolor=WildStrawberry!1,
			nobreak=false
		}
]{thmanswerbox}

\declaretheoremstyle[
	headfont=\bfseries\sffamily\color{Violet!70!black}, bodyfont=\normalfont,
	mdframed={
			linewidth=2pt,
			rightline=false, topline=false, bottomline=false,
			linecolor=Violet, backgroundcolor=Violet!1,
			nobreak=false
		}
]{conjpurplebox}

\declaretheorem[style=thmgreenbox, name=Definition, numberwithin=section]{defn}
\declaretheorem[style=thmgreen2box, name=Definition, numbered=no]{definition*}
\declaretheorem[style=thmredbox, name=Theorem, numberwithin=section]{thm}
\declaretheorem[style=thmred2box, name=Theorem, numbered=no]{theorem*}
\declaretheorem[style=thmredbox, name=Lemma, numberwithin=section]{lmm}
\declaretheorem[style=thmredbox, name=Proposition, numberwithin=section]{prp}
\declaretheorem[style=thmredbox, name=Corollary, numberwithin=section]{crl}
\declaretheorem[style=thmpinkbox, name=Problem, numberwithin=section]{problem}
\declaretheorem[style=thmpinkbox, name=Problem, numbered=no]{problem*}
\declaretheorem[style=thmblue2box, name=Claim, numbered=no]{claim}
\declaretheorem[style=conjpurplebox, name=Conjecture, numberwithin=section]{conjecture}

\renewcommand\theHdefn{\thesection.\arabic {defn}}
\renewcommand\theHthm{\thesection.\arabic {thm}}
\renewcommand\theHlmm{\thesection.\arabic {lmm}}
\renewcommand\theHprp{\thesection.\arabic {prp}}
\renewcommand\theHcrl{\thesection.\arabic {crl}}
\renewcommand\theHproblem{\thesection.\arabic {problem}}
\renewcommand\theHconjecture{\thesection.\arabic {conjecture}}

% Redefine proof environment to get a full control.
\makeatletter
\renewenvironment{proof}[1][\proofname]{\par
	\pushQED{\qed}%
	\normalfont \topsep-2\p@\@plus6\p@\relax
	\trivlist
	\item[\hskip\labelsep
	            \color{RawSienna!70!black}\sffamily\bfseries
	            #1\@addpunct{.}]\ignorespaces
	\begin{mdframed}[linewidth=2pt,rightline=false, topline=false, bottomline=false,linecolor=RawSienna, backgroundcolor=RawSienna!1]
		}{%
		\popQED\endtrivlist\@endpefalse
	\end{mdframed}
}
\makeatother

\declaretheorem[style=thmbluebox, numbered=no, name=Example]{eg}
\declaretheorem[style=thmexplanationbox, numbered=no, name=Proof]{tmpexplanation}
\newenvironment{explanation}[1][]{\vspace{-10pt}\pushQED{\(\circledast\)}\begin{tmpexplanation}}{\null\hfill\popQED\end{tmpexplanation}}

\declaretheorem[style=thmblueline, numbered=no, name=Remark]{remark}
\declaretheorem[style=thmblueline, numbered=no, name=Note]{note}
\declaretheorem[style=thmpinkbox, numbered=no, name=Exercise]{exercise}
\declaretheorem[style=notgrayline, numbered=no, name=As previously seen]{prev}
\declaretheorem[style=thmgreen3box, numbered=no, name=Intuition]{intuition}
\declaretheorem[style=notgraybox, numbered=no, name=Notation]{notation}
\declaretheorem[style=thmanswerbox, numbered=no, name=Answer]{tmpanswer}
\newenvironment{answer}[1][]{\vspace{-10pt}\pushQED{\(\circledast\)}\begin{tmpanswer}}{\null\hfill\popQED\end{tmpanswer}}


\usepackage{etoolbox}
\renewcommand{\qed}{\null\hfill\(\blacksquare\)}

\makeatletter

\def\testdateparts#1{\dateparts#1\relax}
\def\dateparts#1 #2 #3 #4 #5\relax{
	\marginpar{\small\textsf{\mbox{#1 #2 #3 #5}}}
}

\def\@lecture{}%
\newcommand{\lecture}[3]{
	\ifthenelse{\isempty{#3}}{%
		\global\def\@lecture{Lecture #1}%
	}{%
		\global\def\@lecture{Lecture #1: #3}%
	}%
	\section*{\@lecture}
	\marginpar{\small\textsf{\mbox{#2}}}
}

\usepackage{pgffor}%
\newcommand{\lec}[2]{%
	\foreach \c in {#1,...,#2}{%
			\IfFileExists{Lectures/lec_\c.tex} {%
				\input{Lectures/lec_\c.tex}%
			}{}%
		}%
}

% fancy headers
\usepackage{fancyhdr}
\pagestyle{fancy}
\fancyhead[L]{}
\fancyhead[R]{\@lecture}
\fancyfoot[L]{}
\fancyfoot[R]{\thepage}
\fancyfoot[C]{\leftmark}

\makeatother
